\documentclass[UTF8,a4paper,12pt]{ctexbook} 

 \usepackage{graphicx}%学习插入图
 \usepackage{verbatim}%学习注释多行
 \usepackage{booktabs}%表格
 \usepackage{geometry}%图片
 \usepackage{amsmath}
 \usepackage{amssymb}
 \usepackage{listings}%代码
 \usepackage{xcolor}  %颜色
 \usepackage{enumitem}%列表格式
 \usepackage{tcolorbox}
 \usepackage{algorithm}  %format of the algorithm
 \usepackage{algorithmic}%format of the algorithm
 \usepackage{multirow}   %multirow for format of table
 \usepackage{tabularx} 	%表格排版格式控制
 \usepackage{array}	%表格排版格式控制
 \usepackage{hyperref} %超链接 \url{URL}

 \CTEXsetup[format+={\flushleft}]{section}
  %%%% 下面的命令添加新字体 %%%%%
 
 %%%% 下面的命令定义图表、算法、公式 %%%%
 \newcommand{\EQ}[1]{$\textbf{EQ:}#1\ $}
 \newcommand{\ALGORITHM}[1]{$\textbf{Algorithm:}#1\ $}
 \newcommand{\Figure}[1]{$\textbf{Figure }#1\ $}
 
 %%%% 下面命令改变图表下标题的前缀 %%%%% 如:图-1、Fig-1
 \renewcommand{\figurename}{Fig}
 
 \geometry{left=1.6cm,right=1.8cm,top=2cm,bottom=1.7cm} %设置文章宽度
 
 \pagestyle{plain} 		  %设置页面布局

 %代码效果定义
 \definecolor{mygreen}{rgb}{0,0.6,0}
 \definecolor{mygray}{rgb}{0.5,0.5,0.5}
 \definecolor{mymauve}{rgb}{0.58,0,0.82}
 \lstset{ %
 	backgroundcolor=\color{white},   % choose the background color
 	basicstyle=\footnotesize\ttfamily,      % size of fonts used for the code
 	%stringstyle=\color{codepurple},
 	%basicstyle=\footnotesize,
 	%breakatwhitespace=false,         
 	%breaklines=true,                 
 	%captionpos=b,                    
 	%keepspaces=true,                 
 	%numbers=left,                    
 	%numbersep=5pt,                  
 	%showspaces=false,                
 	%showstringspaces=false,
 	%showtabs=false,        
 	columns=fullflexible,
 	breaklines=true,                 % automatic line breaking only at whitespace
 	captionpos=b,                    % sets the caption-position to bottom
 	tabsize=4,
 	commentstyle=\color{mygreen},    % comment style
 	escapeinside={\%*}{*)},          % if you want to add LaTeX within your code
 	keywordstyle=\color{blue},       % keyword style
 	stringstyle=\color{mymauve}\ttfamily,     % string literal style
 	frame=single,
 	rulesepcolor=\color{red!20!green!20!blue!20},
 	% identifierstyle=\color{red},
 	language=c++,
 }
 \author{\kaishu 郑华}
 \title{\heiti Excel 学习笔记}
 
\begin{document}          %正文排版开始
 	\maketitle
	\tableofcontents
	
\chapter{时间}
	\section{相关函数}
		\begin{table}[h]
			\centering
			\begin{tabular}{|m{2cm}|m{6cm}|m{9cm}|}	
				\hline
				 函数名    & 函数说明 & 语法 \\
				
				\hline
				 \verb|DATEIF| 	  & 计算两个日期之间的天数(\verb|"d"|)、月数(\verb|"m"|)、年数(\verb|"y"|) 	  & \verb|DATEDIF(start_date,end_date,unit)|: \verb|DATEDIF("2001/1/1","2003/1/1","Y")|等于2,即时间段中有两个整年\\
				
				\hline
				 \verb|WEEKDAY|	  & 返回某日期为星期几。默认情况下,其值为 1(星期天)到 7(星期六)之间的整数 ,
				 \verb|Return_type|为确定返回值类型的数字,数字 1或省略则 (\verb|1~7|) 代表星期天到数星期六,数字2则(\verb|1~7|)代表星期一到星期天,数字3则(\verb|0~6|)代表 星期一到星期天。
				  &\verb|WEEKDAY(serial_number,return_type)|: \verb|公式 "=WEEKDAY("2001/8/28",2)"|返回2(星期二),\verb|"=WEEKDAY("2003/02/23",3)"|返回6(星期日) \\
				
				\hline
				\verb|WEEKNUM|	  & 返回一个数字,该数字代表一年中的第几周,\verb|Return_type|
				为一数字,确定星期计算从哪一天开始。默认值为1
				&\verb|WEEKNUM(serial_num,return_type)|: \verb|=WEEKNUM("2016/01/01",1) -->1| \\
				\hline
			\end{tabular}
			\caption{常用时间函数}
		\end{table} 
\chapter{画图}

\chapter{表格}

	    
\end{document} 
 		    