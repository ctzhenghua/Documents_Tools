\documentclass[UTF8,a4paper,12pt]{ctexbook} 

\usepackage{graphicx}%学习插入图
\usepackage{verbatim}%学习注释多行
\usepackage{booktabs}%表格
\usepackage{geometry}%图片
\usepackage{amsmath}
\usepackage{amssymb}
\usepackage{listings}%代码
\usepackage{xcolor}  %颜色
\usepackage{enumitem}%列表格式
\setenumerate[1]{itemsep=0pt,partopsep=0pt,parsep=\parskip,topsep=5pt}
\setitemize[1]{itemsep=0pt,partopsep=0pt,parsep=\parskip,topsep=5pt}
\setdescription{itemsep=0pt,partopsep=0pt,parsep=\parskip,topsep=5pt}
\usepackage{tcolorbox}
\usepackage{algorithm}  %format of the algorithm
\usepackage{algorithmic}%format of the algorithm
\usepackage{multirow}   %multirow for format of table
\usepackage{tabularx} 	%表格排版格式控制
\usepackage{array}	%表格排版格式控制
\usepackage{hyperref} %超链接 \url{URL}
\usepackage{tikz}
\usepackage{dirtree}


\usetikzlibrary{intersections,
	positioning,
	petri,
	backgrounds,
	fit,
	decorations.pathmorphing,
	arrows,
	arrows.meta,
	bending,
	calc,
	intersections,
	through,
	backgrounds,
	shapes.geometric,
	quotes,
	matrix,
	trees,
	shapes.symbols,
	graphs,
	math,
	patterns,
	external}
\CTEXsetup[format+={\flushleft}]{section}

%%%% 设置图片目录
\graphicspath{{figure/}}

%%%% 段落首行缩进两个字 %%%%
\makeatletter
\let\@afterindentfalse\@afterindenttrue
\@afterindenttrue
\makeatother
\setlength{\parindent}{2em}  %中文缩进两个汉字位

%%%% 下面的命令重定义页面边距,使其符合中文刊物习惯 %%%%
\addtolength{\topmargin}{-54pt}
\setlength{\oddsidemargin}{0.63cm}  % 3.17cm - 1 inch
\setlength{\evensidemargin}{\oddsidemargin}
\setlength{\textwidth}{14.66cm}
\setlength{\textheight}{24.00cm}    % 24.62

%%%% 下面的命令设置行间距与段落间距 %%%%
\linespread{1.4}
\setlength{\parskip}{0.5\baselineskip}
\geometry{left=1.6cm,right=1.8cm,top=2cm,bottom=1.7cm} %设置文章宽度
\pagestyle{plain} 		  %设置页面布局

%代码效果定义
\definecolor{mygreen}{rgb}{0,0.6,0}
\definecolor{mygray}{rgb}{0.5,0.5,0.5}
\definecolor{mymauve}{rgb}{0.58,0,0.82}
\lstset{ %
	backgroundcolor=\color{white},   % choose the background color
	basicstyle=\footnotesize\ttfamily,      % size of fonts used for the code
	%stringstyle=\color{codepurple},
	%basicstyle=\footnotesize,
	%breakatwhitespace=false,         
	%breaklines=true,                 
	%captionpos=b,                    
	%keepspaces=true,                 
	%numbers=left,                    
	%numbersep=5pt,                  
	%showspaces=false,                
	%showstringspaces=false,
	%showtabs=false,        
	columns=fullflexible,
	breaklines=true,                 % automatic line breaking only at whitespace
	captionpos=b,                    % sets the caption-position to bottom
	tabsize=4,
	commentstyle=\color{mygreen},    % comment style
	escapeinside={\%*}{*)},          % if you want to add LaTeX within your code
	keywordstyle=\color{blue},       % keyword style
	stringstyle=\color{mymauve}\ttfamily,     % string literal style
	frame=single,
	rulesepcolor=\color{red!20!green!20!blue!20},
	% identifierstyle=\color{red},
	language=c++,
}
 \author{\kaishu 郑华}
 \title{\heiti 算法题库笔记}
 
\begin{document}          %正文排版开始
 	\maketitle
\chapter{设置}

\chapter{快捷键}
	\section{调试快捷键}
		\begin{itemize}
			\item \verb|F5 |启动调试
			\item \verb|Shift+F5 |停止调试
			\item \verb|Ctrl+Shift+F9 |删除全部断点
			\item \verb|F10 |逐过程
			\item \verb|F11 |逐语句
		\end{itemize}
	
	\section{编辑快捷键}
		\begin{itemize}
			\item \verb|Shift + Alt + Enter |切换为全屏显示
			\item \verb|F12 |转到调用过程、变量的定义
			\item \verb|Shift + f12 |查找所有引用
			\item \verb|Ctrl + Shift + L |删除当前行
			\item \verb|Ctrl + G |转到指定行
		\end{itemize}
	
	\section{代码快捷键}
		\begin{itemize}
			\item \verb|Ctrl + k d |格式化代码
			\item \verb|Ctrl + e f |格式化选中代码
			\item \verb|Ctrl + k c |注释选定内容
			\item \verb|Ctrl + k u |取消选定内容注释
			\item \verb|Ctrl + j |智能提示
			\item \verb|Ctrl + k p |列出参数信息
			\item \verb|Ctrl + k l |列出成员
			\item \verb|Ctrl + m |折叠或展开当前方法
			\item \verb|Ctrl + m o|折叠所有方法
			\item \verb|Ctrl + m l|展开所有方法
		\end{itemize}
	
	\section{查找快捷键}
		\begin{itemize}
			\item \verb|Ctrl+F |查找
			\item \verb|F3 |查找下一个
			\item \verb|Shift + F3 |查找上一个
			\item \verb|Ctrl + H |替换
		\end{itemize}
	
	\section{项目快捷键}
		\begin{itemize}
			\item \verb|Ctrl + Shift + C |显示类视图窗口
			\item \verb|Ctrl + Shift + S |全部保存
		\end{itemize}
	
	\section{窗口快捷键}  
		\begin{itemize}
			\item \verb|Ctrl + W C |类视图
			\item \verb|Ctrl + W E |错误列表
			\item \verb|Ctrl + W O |输出窗口
			\item \verb|Ctrl + W P |属性窗口
			\item \verb|Ctrl + D B |断点窗口
		\end{itemize}	    
\end{document} 
 		    